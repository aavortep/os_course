\specsection{Введение}

В настоящее время большую актуальность имеют системы, предоставляющие информацию о ресурсах операционной системы и частоте системных вызовов. Имея такие сведения, пользователь может проанализировать состояние системы и нагрузку на неё. Особое внимание уделяется операционным системам с ядром Linux~\cite{linux}. Ядро Linux возможно изучать благодаря тому, что оно имеет открытый исходный код.

На данный момент существует множество различных утилит и команд для получения информации о свободной и занятой оперативной памяти в Linux. Одни из наиболее известных -- это команды free, vmstat, htop, memstat~\cite{commands}. Также существует приложение GNOME System Monitor, предоставляющее краткую статистику использования системных ресурсов -- памяти, процессора, подкачки и сети -- в графическом виде~\cite{gnome}.