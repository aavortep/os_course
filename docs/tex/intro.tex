\specsection{Введение}

В настоящее время большую актуальность имеют системы, предоставляющие информацию о ресурсах операционной системы. Имея такие сведения, пользователь может проанализировать состояние системы и нагрузку на неё. Особое внимание уделяется операционным системам с ядром Linux~\cite{linux}. Ядро Linux возможно изучать благодаря тому, что оно имеет открытый исходный код.

На данный момент существует множество различных утилит и команд для получения информации о свободной и занятой оперативной памяти в Linux. Одни из наиболее известных -- это команды free, vmstat, htop, memstat~\cite{commands}. Также существует приложение GNOME System Monitor, предоставляющее краткую статистику использования системных ресурсов -- памяти, процессора, подкачки и сети -- в графическом виде~\cite{gnome}.

Цель данной работы – реализовать загружаемый модуль ядра Linux, предоставляющий статистику по количеству доступной и занятой оперативной памяти за определенный промежуток времени.

Чтобы достигнуть поставленной цели, требуется решить следующие задачи:
\begin{itemize}
	\item выполнить постановку задачи;
	\item проанализировать и сравнить существующие методы и способы её решения;
	\item описать алгоритм решения поставленной задачи и привести соответствующие схемы и IDEF0-диаграммы;
	\item разработать ПО в соответствии с заданием;
	\item проанализировать результаты работы разработанного ПО.
\end{itemize}