\anonsection{ПРИЛОЖЕНИЕ А}

\begin{lstlisting}[caption={листинг файла monitor\_main.c}]
	#include <linux/module.h>
	#include <linux/proc_fs.h> 
	#include <linux/time.h>
	#include <linux/kthread.h>
	
	#include "memory.h"
	#include "stat.h"
	
	MODULE_LICENSE("GPL");
	MODULE_AUTHOR("Petrova Anna");
	MODULE_DESCRIPTION("A utility for monitoring RAM usage");
	
	static struct proc_dir_entry *proc_root = NULL;
	static struct proc_dir_entry *proc_mem_file = NULL;
	static struct task_struct *worker_task = NULL;
	
	static int show_memory(struct seq_file *m, void *v) {
		print_memory_statistics(m);
		return 0;
	}
	
	static int proc_memory_open(struct inode *sp_inode, struct file *sp_file) {
		return single_open(sp_file, show_memory, NULL);
	}
	
	static int proc_release(struct inode *sp_node, struct file *sp_file) {
		return 0;
	}
	
	mem_info_t mem_info_array[MEMORY_ARRAY_SIZE];
	int mem_info_calls_cnt;
	
	int memory_cnt_task_handler_fn(void *args) {
		struct sysinfo i;
		struct timespec64 t;
		
		ENTER_LOG();
		
		allow_signal(SIGKILL);
		
		while (!kthread_should_stop()) {
			si_meminfo(&i);
			
			ktime_get_real_ts64(&t);
			
			mem_info_array[mem_info_calls_cnt].free = i.freeram;
			mem_info_array[mem_info_calls_cnt].available = si_mem_available();
			mem_info_array[mem_info_calls_cnt++].time_secs = t.tv_sec;
			
			ssleep(10);
			
			if (signal_pending(worker_task)) {
				break;
			}
		}
		
		EXIT_LOG();
		do_exit(0);
		return 0;
	}
	
	#define CHAR_TO_INT(ch) (ch - '0')
	
	static ktime_t convert_strf_to_seconds(char buf[]) {
		/* time format: xxhyymzzs. For example: 01h23m45s */
		ktime_t hours, min, secs;
		
		hours = CHAR_TO_INT(buf[0]) * 10 + CHAR_TO_INT(buf[1]);
		min = CHAR_TO_INT(buf[3]) * 10 + CHAR_TO_INT(buf[4]);
		secs = CHAR_TO_INT(buf[6]) * 10 + CHAR_TO_INT(buf[7]);
		
		return hours * 60 * 60 + min * 60 + secs;
	}
	
	static const struct proc_ops mem_ops = {
		proc_read: seq_read,
		proc_open: proc_memory_open,
		proc_release: proc_release,
	};
	
	static void cleanup(void) {
		ENTER_LOG();
		
		if (worker_task) {
			kthread_stop(worker_task);
		}
		
		if (proc_mem_file != NULL) {
			remove_proc_entry("memory", proc_root);
		}
		
		if (proc_root != NULL) {
			remove_proc_entry(MODULE_NAME, NULL);
		}
		
		EXIT_LOG();
	}
	
	static int proc_init(void) {
		ENTER_LOG();
		
		if ((proc_root = proc_mkdir(MODULE_NAME, NULL)) == NULL) {
			goto err;
		}
		
		if ((proc_mem_file = proc_create("memory", 066, proc_root, &mem_ops)) == NULL) {
			goto err;
		}
		
		EXIT_LOG();
		return 0;
		
		err:
		cleanup();
		EXIT_LOG();
		return -ENOMEM;
	}
	
	static int __init md_init(void) {
		int rc;
		int cpu;
		
		ENTER_LOG();
		
		if ((rc = proc_init())) {
			return rc;
		}
		
		cpu = get_cpu();
		worker_task = kthread_create(memory_cnt_task_handler_fn, NULL, "memory counter thread");
		kthread_bind(worker_task, cpu);
		
		if (worker_task == NULL) {
			cleanup();
			return -1;
		}
		
		wake_up_process(worker_task);
		
		printk("%s: module loaded\n", MODULE_NAME);
		EXIT_LOG();
		
		return 0;
	}
	
	static void __exit md_exit(void) { 
		cleanup();
		
		printk("%s: module unloaded\n", MODULE_NAME); 
	}
	
	module_init(md_init);
	module_exit(md_exit);
\end{lstlisting}

\begin{lstlisting}[caption={листинг файла stat.c}]
	#include "stat.h"
	
	#define TASK_STATE_FIELD state
	#define TASK_STATE_SPEC "%ld"
	
	static inline long convert_to_kb(const long n) {
		return n << (PAGE_SHIFT - 10);
	}
	
	void print_memory_statistics(struct seq_file *m) {
		struct sysinfo info;
		long long secs;
		long sys_occupied, apps_occupied;
		int i;
		
		ENTER_LOG();
		
		si_meminfo(&info);
		show_int_message(m, "Memory total: \t%ld kB\n", convert_to_kb(info.totalram));
		
		for (i = 0; i < mem_info_calls_cnt; i++) {
			secs = mem_info_array[i].time_secs;
			show_int3_message(m, "\nTime %.2llu:%.2llu:%.2llu\n", (secs / 3600 + 3) % 24, secs / 60 % 60, secs % 60);
			show_int_message(m, "Free:      \t\t\t%ld kB\n", convert_to_kb(mem_info_array[i].free));
			show_int_message(m, "Available: \t\t\t%ld kB\n", convert_to_kb(mem_info_array[i].available));
			sys_occupied = convert_to_kb(info.totalram) - convert_to_kb(mem_info_array[i].available);
			show_int_message(m, "Occupied: \t%ld kB\n", sys_occupied);
		}
		
		EXIT_LOG();
	}
\end{lstlisting}

\newpage

\begin{lstlisting}[caption={листинг файла log.c}]
	#include "log.h"
	
	void show_int_message(struct seq_file *m, const char *const f, const long num) {
		char tmp[256];
		int len;
		
		len = snprintf(tmp, 256, f, num);
		seq_write(m, tmp, len);
	}
	
	void show_int3_message(struct seq_file *m, const char *const f, const long n1, const long n2, const long n3) {
		char tmp[256];
		int len;
		
		len = snprintf(tmp, 256, f, n1, n2, n3);
		seq_write(m, tmp, len);
	}
	
	void show_str_message(struct seq_file *m, const char *const f, const char *const s) {
		char tmp[256];
		int len;
		
		len = snprintf(tmp, 256, f, s);
		seq_write(m, tmp, len);
	}
\end{lstlisting}

\begin{lstlisting}[caption={листинг файла memory.h}]
	#ifndef __MEMORY_H__
	#define __MEMORY_H__
	
	#include <linux/kthread.h>
	#include <linux/delay.h>
	#include <linux/time.h>
	
	#include "log.h"
	
	typedef struct mem_struct {
		long available;
		long free;
		long time_secs;
	} mem_info_t;
	
	#define MEMORY_ARRAY_SIZE 8640
	extern mem_info_t mem_info_array[MEMORY_ARRAY_SIZE];
	
	extern int mem_info_calls_cnt;
	
	#endif
\end{lstlisting}
