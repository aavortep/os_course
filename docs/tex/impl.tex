\section{Технологический раздел}

\subsection{Выбор языка и среды программирования}

В качестве языка программирования был выбран язык C~\cite{c99}. Данный выбор обусловлен тем, что исходный код ядра Linux, все его модули и драйверы написаны на этом языке.

В качестве компилятора был выбран gcc~\cite{gcc}, а в качестве среды программирования -- QtCreator.

\subsection{Информация о памяти в системе}

Для сбора информации об оперативной памяти в системе запускается отдельный поток ядра, который находится в состоянии сна и, просыпаясь каждую секунду, фиксирует эту информацию в результирующий массив. В листинге \ref{lst:thread} представлена реализация этого потока, а в листинге \ref{lst:threadinit} его инициализация.

\begin{lstlisting}[label={lst:thread}, caption={реализация функции сохраняющей информацию о памяти}]
	mem_info_t mem_info_array[MEMORY_ARRAY_SIZE];
	int mem_info_calls_cnt;
	
	int memory_cnt_task_handler_fn(void *args) {
		struct sysinfo i;
		struct timespec64 t;
		
		ENTER_LOG();
		
		allow_signal(SIGKILL);
		
		while (!kthread_should_stop()) {
			si_meminfo(&i);
			
			ktime_get_real_ts64(&t);
			
			mem_info_array[mem_info_calls_cnt].free = i.freeram;
			mem_info_array[mem_info_calls_cnt].available = si_mem_available();
			mem_info_array[mem_info_calls_cnt].shared = i.sharedram;
			mem_info_array[mem_info_calls_cnt].buffers = i.bufferram;
			mem_info_array[mem_info_calls_cnt].swap = i.totalswap;
			mem_info_array[mem_info_calls_cnt].totalhigh = i.totalhigh;
			mem_info_array[mem_info_calls_cnt].freehigh = i.freehigh;
			mem_info_array[mem_info_calls_cnt++].time_secs = t.tv_sec;
			
			ssleep(1);
			
			if (signal_pending(worker_task)) {
				break;
			}
		}
		
		EXIT_LOG();
		do_exit(0);
		return 0;
	}
\end{lstlisting}

\begin{lstlisting}[label={lst:threadinit}, caption={инициализация потока ядра}]
	cpu = get_cpu();
	worker_task = kthread_create(memory_cnt_task_handler_fn, NULL, "memory counter thread");
	kthread_bind(worker_task, cpu);
	
	if (worker_task == NULL) {
		cleanup();
		return -1;
	}
	
	wake_up_process(worker_task);
	return 0;
\end{lstlisting}

\subsection{Поиск адреса перехватываемой функции}

Для корректной работы ftrace необходимо найти и сохранить адрес функции, которую будет перехватывать разрабатываемый модуль ядра. 

Начиная с версии ядра 5.7.0 функция kallsyms\_lookup\_name()~\cite{kallsyms-removed}, позволяющая найти адрес перехватываемой функции, перестала быть экспортируемой. Так как модуль ядра разрабатывался на системе с версией ядра 5.15.0, для поиска адреса перехватываемой функции использовался интерфейс kprobes.

\begin{lstlisting}[label=lst:lookup_name, caption=Реализация функции lookup\_name()]
	#if LINUX_VERSION_CODE >= KERNEL_VERSION(5,7,0)
	static unsigned long lookup_name(const char *name)
	{
		struct kprobe kp = {
			.symbol_name = name
		};
		unsigned long retval;
		
		ENTER_LOG();
		
		if (register_kprobe(&kp) < 0) {
			EXIT_LOG();
			return 0;
		}
		
		retval = (unsigned long) kp.addr;
		unregister_kprobe(&kp);
		
		EXIT_LOG();
		
		return retval;
	}
	#else
	static unsigned long lookup_name(const char *name)
	{
		unsigned long retval;
		
		ENTER_LOG();
		retval = kallsyms_lookup_name(name);
		EXIT_LOG();
		
		return retval;
	}
	#endif
\end{lstlisting}

\subsection{Инициализация ftrace}

В листинге \ref{lst:install_hook} представлена реализация функции, которая инициализирует структуру ftrace\_ops.

\begin{lstlisting}[label=lst:install_hook, caption=Реализация функции install\_hook()]
	static int install_hook(struct ftrace_hook *hook) {
		int rc;
		
		ENTER_LOG();
		
		if ((rc = resolve_hook_address(hook))) {
			EXIT_LOG();
			return rc;
		}
		
		hook->ops.func = ftrace_thunk; 
		hook->ops.flags = FTRACE_OPS_FL_SAVE_REGS
		| FTRACE_OPS_FL_RECURSION
		| FTRACE_OPS_FL_IPMODIFY;
		
		if ((rc = ftrace_set_filter_ip(&hook->ops, hook->address, 0, 0))) {
			pr_debug("ftrace_set_filter_ip() failed: %d\n", rc);
			return rc;
		}
		
		if ((rc = register_ftrace_function(&hook->ops))) {
			pr_debug("register_ftrace_function() failed: %d\n", rc);
			ftrace_set_filter_ip(&hook->ops, hook->address, 1, 0);
		}
		
		EXIT_LOG();
		
		return rc;
	}
\end{lstlisting}

В листинге \ref{lst:remove_hook} представлена реализация отключения перехвата функции.

\begin{lstlisting}[label=lst:remove_hook, caption=Реализация функции remove\_hook()]
	static void remove_hook(struct ftrace_hook *hook) {
		int rc;
		
		ENTER_LOG();
		
		if (hook->address == 0x00) {
			EXIT_LOG();
			return;
		}
		
		if ((rc = unregister_ftrace_function(&hook->ops))) {
			pr_debug("unregister_ftrace_function() failed: %d\n", rc);
		}
		
		if ((rc = ftrace_set_filter_ip(&hook->ops, hook->address, 1, 0))) {
			pr_debug("ftrace_set_filter_ip() failed: %d\n", rc);
		}
		
		hook->address = 0x00;
		
		EXIT_LOG();
	}
\end{lstlisting}

\subsection{Функции обёртки}

При объявлении функций обёрток, которые будут запущены вместо перехватываемой функции, необходимо в точности соблюдать сигнатуру. Оригинальные описания функций были взяты из исходных кодов ядра Linux. 

В листинге \ref{lst:sys_execve} представлена реализация функции обёртки на примере sys\_clone().

\begin{lstlisting}[label=lst:sys_execve, caption=Реализация функции обёртки]
	static asmlinkage long (*real_sys_clone)(unsigned long clone_flags,
	unsigned long newsp, int __user *parent_tidptr,
	int __user *child_tidptr, unsigned long tls);
	
	static asmlinkage long hook_sys_clone(unsigned long clone_flags,
	unsigned long newsp, int __user *parent_tidptr,
	int __user *child_tidptr, unsigned long tls)
	{
		update_syscall_array(SYS_CLONE_NUM);
		return real_sys_clone(clone_flags, newsp, parent_tidptr, child_tidptr, tls);
	}
\end{lstlisting}

В листинге \ref{lst:update_syscall_array} представлена реализация функции которая обновляет массив, хранящий количество системных вызовов за последние 24 часа.

\begin{lstlisting}[label=lst:update_syscall_array, caption=Реализация функции update\_syscall\_array()]
	static DEFINE_SPINLOCK(my_lock);
	
	static void inline update_syscall_array(int syscall_num) {
		ktime_t time;
		
		time = ktime_get_boottime_seconds() - start_time;
		
		spin_lock(&my_lock);
		
		if (syscall_num < 64) {
			syscalls_time_array[time % TIME_ARRAY_SIZE].p1 |= 1UL << syscall_num;
		} else {
			syscalls_time_array[time % TIME_ARRAY_SIZE].p2 |= 1UL << (syscall_num % 64);
		}
		
		spin_unlock(&my_lock);
	}
\end{lstlisting}

\subsection{Получение информации о количестве системных вызовов}

В листинге \ref{lst:syscall_proc} представлена реализация функций, которые агрегируют информацию о системных вызовах (данные массива update\_syscall\_array) и предоставляют ее в читаемом для пользователя виде.

\begin{lstlisting}[label=lst:syscall_proc, caption=Реализация функций агрегации данных о системных вызовах, language=c]
	static inline void walk_bits_and_find_syscalls(struct seq_file *m, uint64_t num, int syscalls_arr_cnt[]) {
		int i;
		
		for (i = 0; i < 64; i++) {
			if (num & (1UL << i)) {
				syscalls_arr_cnt[i]++;
			}
		}
	}
	
	void print_syscall_statistics(struct seq_file *m, const ktime_t mstart, ktime_t range) {
		int syscalls_arr_cnt[128];
		uint64_t tmp;
		size_t i;
		ktime_t uptime;
		
		memset((void*)syscalls_arr_cnt, 0, 128 * sizeof(int));
		uptime = ktime_get_boottime_seconds() - mstart;
		
		if (uptime < range) {
			range = uptime;
		}
		
		for (i = 0; i < range; i++) {
			if ((tmp = syscalls_time_array[uptime - i].p1) != 0) {
				walk_bits_and_find_syscalls(m, tmp, syscalls_arr_cnt);
			}
			
			if ((tmp = syscalls_time_array[uptime - i].p2) != 0) {
				walk_bits_and_find_syscalls(m, tmp, syscalls_arr_cnt + 64);
			}
		}
		
		show_int_message(m, "Syscall statistics for the last %d seconds.\n\n", range);
		
		for (i = 0; i < 128; i++) {
			if (syscalls_arr_cnt[i] != 0) {
				show_str_message(m, "%s called ", syscalls_names[i]);
				show_int_message(m, "%d times.\n", syscalls_arr_cnt[i]);
			}
		}
	}
\end{lstlisting}

\subsection{Детали реализации}

В листингах \ref{lst:mdinit}-\ref{lst:cleanup} представлена реализация точек входа в загружаемый модуль.

\begin{lstlisting}[label={lst:mdinit}, caption={функция инициализации модуля}]
	static int __init md_init(void) {
		int rc;
		int cpu;
		
		ENTER_LOG();
		
		if ((rc = proc_init())) {
			return rc;
		}
		
		if ((rc = install_hooks())) {
			cleanup();
			return rc;
		}
		
		start_time = ktime_get_boottime_seconds();
		
		cpu = get_cpu();
		worker_task = kthread_create(memory_cnt_task_handler_fn, NULL, "memory counter thread");
		kthread_bind(worker_task, cpu);
		
		if (worker_task == NULL) {
			cleanup();
			return -1;
		}
		
		wake_up_process(worker_task);
		
		printk("%s: module loaded\n", MODULE_NAME);
		EXIT_LOG();
		
		return 0;
	}
\end{lstlisting}

\begin{lstlisting}[label={lst:mdexit}, caption={функция выхода из модуля}]
	static void __exit md_exit(void) { 
		cleanup();
		
		printk("%s: module unloaded\n", MODULE_NAME); 
	}
\end{lstlisting}

\begin{lstlisting}[label={lst:cleanup}, caption={функция cleanup()}]
	static void cleanup(void) {
		ENTER_LOG();
		
		if (worker_task) {
			kthread_stop(worker_task);
		}
		
		if (proc_mem_file != NULL) {
			remove_proc_entry("memory", proc_root);
		}
		
		if (proc_syscall_file != NULL) {
			remove_proc_entry("syscalls", proc_root);
		}
		
		if (proc_root != NULL) {
			remove_proc_entry(MODULE_NAME, NULL);
		}
		
		remove_hooks();
		
		EXIT_LOG();
	}
\end{lstlisting}

Make файл для компиляции и сборки разработанного загружаемого модуля ядра представлен в листинге \ref{lst:makefile}.

\begin{lstlisting}[label={lst:makefile}, caption={реализация make файла}]
	KPATH := /lib/modules/$(shell uname -r)/build
	MDIR := $(shell pwd)
	
	obj-m += monitor.o
	monitor-y := monitor_main.o stat.o hooks.o log.o
	EXTRA_CFLAGS=-I$(PWD)/inc
	
	all:
	make -C $(KPATH) M=$(MDIR) modules
	
	clean:
	make -C $(KPATH) M=$(MDIR) clean
	
	load:
	sudo insmod monitor.ko
	
	unload:
	sudo rmmod monitor.ko
	
	info:
	modinfo monitor.ko
	
	logs:
	sudo dmesg | tail -n60 | grep monitor:
\end{lstlisting}
