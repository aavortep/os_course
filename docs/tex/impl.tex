\section{Технологическая часть}

\subsection{Выбор языка программирования}

В качестве языка программирования был выбран язык C~\cite{c99}. Данный выбор обусловлен тем, что исходный код ядра Linux, все его модули и драйверы написаны на этом языке.

В качестве компилятора был выбран gcc~\cite{gcc}.

\subsection{Информация о памяти в системе}

Для сбора информации о доступной и свободной памяти в системе запускается отдельный поток ядра, который находится в состоянии сна и,
просыпаясь каждые 10 секунд, фиксирует эту информацию в результирующий массив. В листинге \ref{lst:thread} представлена реализация этого потока, а в листинге \ref{lst:threadinit} его инициализация.

\begin{lstlisting}[label={lst:thread}, caption={реализация функции сохраняющей информацию о памяти}]
	mem_info_t mem_info_array[MEMORY_ARRAY_SIZE];
	int mem_info_calls_cnt;
	
	int memory_cnt_task_handler_fn(void *args) {
		struct sysinfo i;
		struct timespec64 t;
		
		ENTER_LOG();
		
		allow_signal(SIGKILL);
		
		while (!kthread_should_stop()) {
			si_meminfo(&i);
			
			ktime_get_real_ts64(&t);
			
			mem_info_array[mem_info_calls_cnt].free = i.freeram;
			mem_info_array[mem_info_calls_cnt].available = si_mem_available();
			mem_info_array[mem_info_calls_cnt++].time_secs = t.tv_sec;
			
			ssleep(10);
			
			if (signal_pending(worker_task)) {
				break;
			}
		}
		
		EXIT_LOG();
		do_exit(0);
		return 0;
	}
\end{lstlisting}

\begin{lstlisting}[label={lst:threadinit}, caption={инициализация потока ядра}]
	cpu = get_cpu();
	worker_task = kthread_create(memory_cnt_task_handler_fn, NULL, "memory counter thread");
	kthread_bind(worker_task, cpu);
	
	if (worker_task == NULL) {
		cleanup();
		return -1;
	}
	
	wake_up_process(worker_task);
	return 0;
\end{lstlisting}

\subsection{Детали реализации}

В листингах \ref{lst:mdinit}-\ref{lst:procinit} представлена реализация точек входа в загружаемый модуль.

\begin{lstlisting}[label={lst:mdinit}, caption={функция инициализации модуля}]
	static int __init md_init(void) {
		int rc;
		int cpu;
		
		ENTER_LOG();
		
		if ((rc = proc_init())) {
			return rc;
		}
		
		cpu = get_cpu();
		worker_task = kthread_create(memory_cnt_task_handler_fn, NULL, "memory counter thread");
		kthread_bind(worker_task, cpu);
		
		if (worker_task == NULL) {
			cleanup();
			return -1;
		}
		
		wake_up_process(worker_task);
		
		printk("%s: module loaded\n", MODULE_NAME);
		EXIT_LOG();
		
		return 0;
	}
\end{lstlisting}

\begin{lstlisting}[label={lst:mdexit}, caption={функция выхода из модуля}]
	static void __exit md_exit(void) { 
		cleanup();
		
		printk("%s: module unloaded\n", MODULE_NAME); 
	}
\end{lstlisting}

\begin{lstlisting}[label={lst:procinit}, caption={создание директории и файла в /proc}]
	static int proc_init(void) {
		ENTER_LOG();
		
		if ((proc_root = proc_mkdir(MODULE_NAME, NULL)) == NULL) {
			goto err;
		}
		
		if ((proc_mem_file = proc_create("memory", 066, proc_root, &mem_ops)) == NULL) {
			goto err;
		}
		
		EXIT_LOG();
		return 0;
		
	err:
		cleanup();
		EXIT_LOG();
		return -ENOMEM;
	}
\end{lstlisting}

Make файл для компиляции и сборки разработанного загружаемого модуля ядра представлен в листинге \ref{lst:makefile}.

\begin{lstlisting}[label={lst:makefile}, caption={реализация make файла}]
	KPATH := /lib/modules/$(shell uname -r)/build
	MDIR := $(shell pwd)
	
	obj-m += monitor.o
	monitor-y := monitor_main.o stat.o log.o
	EXTRA_CFLAGS=-I$(PWD)/inc
	
	all:
		make -C $(KPATH) M=$(MDIR) modules
	
	clean:
		make -C $(KPATH) M=$(MDIR) clean
	
	load:
		sudo insmod monitor.ko
	
	unload:
		sudo rmmod monitor.ko
	
	info:
		modinfo monitor.ko
	
	logs:
		sudo dmesg | tail -n60 | grep monitor:
\end{lstlisting}

\subsection*{Выводы}

В данном разделе был выбран язык программирования, разработан и протестирован исходный код программы, а также рассмотрены листинги реализованного программного обеспечения.
